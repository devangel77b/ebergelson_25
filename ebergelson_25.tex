\documentclass[12pt]{letter}

% to sign: gpg --local-user 0x0037AFD46556700F --clearsign --output=kchen_25_evangelista_signed.pdf --not-dash-escaped kchen_25_evangelista.pdf
% to verify: gpg --verify kchen_25_evangelista_signed.pdf

\usepackage{mbsletter}
\usepackage{siunitx}

\newcommand\firstname{Eric}
\newcommand\lastname{Bergelson}
\newcommand\subject{he}
\newcommand\object{him}
\newcommand\possessive{his}
\newcommand\reflexive{himself}
\newcommand\adjective{happy}

\title{Recommendation for \firstname\ \lastname}
\author{Dennis Evangelista}
% For letters of recommendation, MBS asks that you do not date your letter
\date{} 
%\date{\today}
%\usepackage[american,inputamerican]{isodate}
%\date{\printdate{6/17/2021}}

% This also sets the PDF metadata so it is searchable in like Spotlight etc. 
\hypersetup{
pdfauthor={Dennis Evangelista},
pdftitle={Recommendation for \firstname\ \lastname},
pdfkeywords={\firstname\ \lastname, Morristown-Beard School, Morristown Beard School, MBS, recommendation}}

% for letter closing use this if you wish to sign hardcopy
\usepackage{designature}
\digitalsignature{\includesignature}
\name{Dennis J.~Evangelista, Ph.D.}

\begin{document}

\begin{letter}{% recipient address here (optional, for future envelope use)
%Massachusetts Institute of Technology\\
%Office of Admissions\\
%77 Massachusetts Avenue, Room E38-200\\
%Cambridge, MA 02139
}

% opening here
\opening{Recommendation for {\scshape\firstname\ \lastname}:}
%\raggedright % if you like this sort of thing
%\setlength{\parindent}{15pt} % if you like this sort of thing

% Keep letter to one page (adjust margins and font size if necessary)
I am happy to recommend \firstname\ \lastname\ for admission to your program.  For three years I have had the pleasure of watching Eric grow as a student in both academic and cocurricular settings at Morristown Beard School. Eric was my student for 9th grade Physics, 11th grade Drone Engineering, and 11th grade Biology. Additionally, Eric was an active member of the Drone Club for which I served as an advisor.  I am a PhD biologist and former US Navy lieutenant and former assistant professor of engineering. 

In biology, Eric has shown a continuously increasing grade each biology test he has taken.  He is interested in the material, asks questions, comes for extra help when needed, and answers well on tests. His lighthearted competition with his friend has improved both of their grades. I was also impressed with Eric's presentation of a peer-reviewed journal article on penguins swimming biomechanics. He was able to distill a complex paper on penguin swimming into something digestible by his classmates.

In drone engineering, Eric worked well with a partner on the design of an underwater remotely operated vehicle based on the design of the SeaPerch ROV.  Eric led his team in developing and testing underwater thrusters using DC motors; his design was modular and able to be mounted to a variety of stock ROV frames based on PVC pipe. Eric led testing of these thrusters in both fresh and salt water, leading up to a long term stress test in which the thrusters were immersed in salt water for up to three days, showing his design would function during a standard mission. He is clearly able to plan and execute a basic engineering study to develop and qualify a design. As another example of Eric's creativity and interests, he used the drones in class to film himself dressed as Thomas Jefferson, threatening King George, for his AP US History class. 

Eric is a culturally-rich young man. He has a strong interest in languages, and respects people along with cultural sensitivities. Eric speaks of his travel experiences with a deep appreciation for traditions of others. Eric even shared a Russian cartoon (Nu Pogodi) with students in class while we were working on soldering airframe electronics. 

Eric has the requisite maturity to succeed in college, and has shown achievement in many different areas in the classes I have taught. I believe Eric would be a great addition to your program. 

% closing here
%\closing{Respectfully,}
\noclosing

%\ps{post script here}
%\encl{enclosure here}
\end{letter}
\end{document}

% From form to function, I enjoy watching ideas come into being. It is what inspires me to grow my toolbox of skills in science research. My interests are in the application of technologies to improve human life, especially as it relates to environmentalism. I participated in a two-week Marine Biology experience focused on the development of systems that help monitor coastal environments. I have also participated in summer studies in Industrial Design, Biomedical Engineering, and Drone Engineering, and have learned to use CAD for architectural design. Understanding how builders create structures to fit in with their surroundings interests me. As a member of the program, I look forward to joining my host community in putting these skills to use. Cultural migration of ideas is a catalyst for human advancement. While sharing my training, I will add to my toolbox, learning local customs in agriculture, and animal husbandry, and observing how community structures reflect cultural practices.
%I am especially looking forward to becoming part of both the local community and my volunteer community. I love bridging my experiences with those of others and have discovered that I learn by gaining new perspectives when I am teaching others. I enjoyed volunteering in the summers at my middle school, teaching math to 6-8-year-olds, and running sports clinics focused on teamwork. I also volunteer at my synagogue’s Youth services, running holiday fairs. I am excited to bring my love of play to my Tanzanian host community. 
%As a lover of the outdoors, I am looking forward to the excursion to Mt. Kilimanjaro and the Safari and Maasai Lands outing. It is a privilege to see the natural wonders of the world, and it is knowledge I hope to share as I continue my environmental work. 
%
%
%
%
%CLUB: Vice President of Movies and TV Club, member of BFI (Business Financial Investment Club), Jewish Affinity Club, and member of Drones Club 
%
%SERVICE: Table of Hope (prepare and serve food for those who need it), Volunteered as a counselor at a children’s summer camp at Chatham Day School, and helped organize events for children at Cedar Lake Summer Camp 
%
%Summer classes: Forensic Science, Woodworking, Saxophone lessons, Marine Biology, Sailing, Civil Engineering, Biomedical Engineering, Industrial Design, and Brazilian Jiu Jitsu
%
%LICENSES: Provisional Drivers License, Open Water Scuba Diving License, Sailing License, and going for my advanced open water license for scuba diving this summer 
%
%SKILLS: Woodworking, Kayaking, Working with Children, Teaching Children, Car repair, Sailing, Scuba Diving, Drone Design, History knowledge, and playing the Saxophone 
%
%LANGUAGES: Fluent in English, able to speak and understand very basic Spanish, Can have a basic conversation in Russian, and able to understand Russian on an advanced level